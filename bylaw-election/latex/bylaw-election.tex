% arara: uplatex until !found('log', 'undefined references')
% arara: dvipdfmx


\documentclass[dvipdfmx, uplatex, 12pt]{jsarticle}

\usepackage{comment}
\usepackage{graphicx}
\usepackage{url}
\usepackage{booktabs}
\usepackage{multirow}
\usepackage{amsmath, amssymb}
\usepackage{spverbatim}

\usepackage[top=20truemm,bottom=25truemm,textwidth=42zw]{geometry}

\begin{document}

\begin{center}
    \begin{Large}
        全学学類・専門学群・総合学域群代表者会議 \\
        選挙細則
    \end{Large}
\end{center}

\begin{flushright}
    2024年3月21日

    全学学類・専門学群・総合学域群代表者会議 \; 決議
\end{flushright}

\vskip 2\baselineskip

\subsection*{第1章 総則}

\noindent
(目的)

第1条 \; この細則は、全学学類・専門学群・総合学域群代表者会議(以下「全代会」という。)において、議長を定める選挙(以下「議長選挙」という。)、副議長を定める選挙(以下「副議長選挙」という。)その他の選挙を執行する際に、その公正さを保証するための規則を定めるものである。

\vskip\baselineskip

\noindent
(選挙管理委員会)

第2条 \; 全代会に選挙管理委員会を置く。

2 \;
選挙管理委員会は、全代会における選挙の実施及び監督を行う。

3 \;
選挙管理委員会は、監察役がこれを監督する。

4 \;
選挙管理委員会がその任務を遂行できないときは、監察役がこれを代行する。

\vskip\baselineskip

\noindent
(選挙管理委員)

第3条 \; 選挙管理委員会は、監察役が指名する委員(以下「選挙管理委員」という。)によって構成される。

2 \;
前項に定める選挙管理委員の指名は、正当な事由のあるとき、これを拒むことができる。

3 \;
選挙管理委員は、次年度の委員が選出されたとき、離任する。

4 \;
選挙管理委員は、議長又は副議長の候補者となることができない。

\vskip\baselineskip

\noindent
(選挙管理委員長)

第4条 \; 選挙管理委員会に、選挙管理委員長を1名置く。

2 \;
選挙管理委員長は、選挙管理委員会を代表して選挙の執行を行い、本細則に不明な点がある際にはこれを補足する。

3 \;
選挙管理委員長は、選挙管理委員の互選によってこれを選出する。

\vskip\baselineskip

\noindent
(選挙管理委員会の所管する選挙)

第5条 \; 選挙管理委員会は、議長選挙及び副議長選挙を所管する。

2 \;
前項の定めにかかわらず、選挙管理委員会は、議長の依頼のあるとき、全代会における選挙を所管する。

\vskip\baselineskip

\noindent
(選挙事務の実施)

第6条 \; 選挙管理委員会の所管する選挙における、投票用紙の作成、開票、集計その他の事務は、選挙管理委員会がこれを行う。

\vskip\baselineskip

\noindent
(自由意志の保護原則)

第7条 \; 選挙管理委員会の所管する選挙は、すべて秘密投票及び平等選挙の原則を保障し、選挙人の自由意志への干渉や情報の対称性を破るいかなる試みも排除せねばならない。

2 \;
前項の目的を達すべく、選挙管理委員は投票用紙の原本やそれに相当する物を、選挙管理委員長の同意なしには絶対に公開してはならない。

3 \;
本条第1項の目的を達すべく、選挙管理委員会の会議は議題とする選挙の候補者に公開してはならない。

\vskip\baselineskip

\noindent
(選挙の告示)

第8条 \; 選挙管理委員会は、選挙の期日の7日前までに、選挙の方法その他選挙に関し必要と認める事項を公示しなければならない。

\vskip\baselineskip

\subsection*{第2章 議長選挙}

\noindent
(議長選挙の実施)

第9条 \; 議長選挙の選挙人は、全ての議員とする。

2 \;
議長選挙は、本会議においてこれを行う。

3\;
議長選挙を議題とする本会議において、議長の選出が完了しないとき、これを休会し、異なる期日に再開する。

4 \;
議長選挙は、選挙人の3分の2以上の投票がなければ成立しない。

5 \;
議長選挙において、不在者投票及び代理投票は、これを認めない。

\vskip\baselineskip

\noindent
(議長への立候補)

第10条 \; 議長の候補者となろうとする者は、選挙の期日の3日前までに、文書でその旨を選挙管理委員長に届け出なければならない。

2 \;
前条第3項に定める休会が発生したとき、選挙管理委員会は追加の立候補の届け出を受け付けることができる。

\vskip\baselineskip

\noindent
(議長選挙における信任投票)

第11条 \; 信任投票では、1名の候補者について、選挙人の総意による信任の是非を問う。

2 \;
信任投票において、選挙人は、信任又は不信任のいずれかに投票する。

3 \;
信任投票において、信任が選挙人の過半数のとき、候補者を選挙人の総意により信任されたものと見做す。

4 \;
信任投票において、不信任が選挙人の過半数のとき、候補者は、当該選挙へ再度立候補することはできない。

\vskip\baselineskip

\noindent
(議長選挙における決選投票)

第12条 \; 決選投票では、2名の候補者から1名を選出する。

2 \;
決選投票により選出された候補者は、これを選挙人の総意により信任されたものと見做す。


3 \;
決選投票において、選挙人は、いずれかの候補者1名に投票する。


4 \;
決選投票において、最も多い得票数の候補者は、これを当選人とする。

\vskip\baselineskip

\noindent
(議長選挙の方法)

第13条 \; 議長選挙は、議長の候補者が1名のとき、信任投票によりこれを行う。

2 \;
議長選挙は、議長の候補者が2名のとき、決選投票によりこれを行う。

\vskip\baselineskip

第14条 \; 議長選挙において、議長の候補者が3名以上のとき、単記による投票を行う。

2 \;
前項に定める投票において、選挙人の過半数の得票者がいるとき、これを当選人とする。

3 \;
第1項に定める投票において、選挙人の過半数の得票者がいないとき、上位得票者2名による決選投票により議長を選出する。

\vskip\baselineskip

\noindent
(同数の得票)

第15条 \; 投票により候補者の順位を決する必要のある場合に、これらの得票数が同数のとき、これらの順位は抽選又は予選投票により決する。

\vskip\baselineskip

\noindent
(無効投票)

第16条 \; 次の各号のいずれかに該当するものは、無効とする。

\begin{enumerate}
    \item 所定の書式に反するもの
    \item 候補者を記載する投票において、一投票中に規定を超える数の候補者を記載したもの
    \item 候補者を記載する投票において、被選挙権のない候補者を記載したもの
    \item 記載する必要のある内容のほか他事を記載したもの。ただし、候補者を記載する投票において、所属、身分又は敬称の類を記入したものは、この限りでない。
    \item 記載内容を確認し難いもの
    \item 白票
\end{enumerate}

\vskip\baselineskip

\subsection*{第3章 副議長選挙}

\noindent
(副議長選挙)

第17条 \; 副議長選挙は、第9条から第12条、第15条及び第16条に準じてこれを行う。

\vskip\baselineskip

\noindent
(副議長選挙の方法)

第18条 \; 副議長選挙は、副議長の候補者が定員以下のとき、各候補者についての信任投票によりこれを行う。

2 \;
議長選挙は、本会議においてこれを行う。

\vskip\baselineskip

第19条 \; 副議長選挙において、副議長の候補者が定員以上のとき、各選挙人は定員と同数の異なる候補者に投票する。

2 \;
前項に定める投票において、選挙人の過半数の得票者がいるとき、これを当選人とする。

3 \;
第1項に定める投票において、選挙人の過半数の得票者がいないとき、上位得票者2名による決選投票により副議長1名を選出する。

4 \;
前2項に定める手続きにおいて定員が満たされないとき、前2項に定める手続きにおいて選出されなかった者のうち第1項に定める投票における上位得票者2名による決選投票により副議長1名を選出する。

\vskip\baselineskip

\subsection*{第4章 雑則}

\noindent
(疑義の解決)

第20条 \; 選挙管理委員長は、選挙管理委員会の所管する選挙において、5名以上の選挙人若しくは1名以上の候補者が選挙の手続きに重大な疑義を呈し又は選挙管理委員会が選挙の手続きに疑義を呈したとき、これの解決を目的として次の各号のいずれか1つ以上の手続きを行う。

\begin{enumerate}
    \item 選挙管理委員及び疑義を呈した者の立会いの下、選挙管理委員長及び選挙管理委員長が指名した者による開票、集計その他の事務の再度の実施
    \item 選挙管理委員長及び選挙管理委員長が指名した選挙管理委員による、選挙事務の代行
    \item 疑義を生じさせた者の当該選挙における諸資格の停止
    \item 選挙の一部又は全部のやり直しについての宣言
    \item 監察役による選挙管理委員会の任務の代行
    \item 選挙管理委員長の信任投票の実施
\end{enumerate}

2 \;
前項の手続きにおいて疑義が解決できないとき、前項の定めにかかわらず選挙管理委員長が必要かつ適切と判断した手段を実行する。

\section*{附則}

本細則は、2024年4月1日から施行する。

\end{document}
