% arara: uplatex until !found('log', 'undefined references')
% arara: dvipdfmx


\documentclass[dvipdfmx, uplatex, 12pt]{jsarticle}

\usepackage{comment}
\usepackage{graphicx}
\usepackage{url}
\usepackage{booktabs}
\usepackage{multirow}
\usepackage{amsmath, amssymb}
\usepackage{spverbatim}

\usepackage[top=20truemm,bottom=25truemm,textwidth=42zw]{geometry}

\begin{document}

\begin{center}
    \begin{Large}
        筑波大学 新歓活動規程
    \end{Large}
\end{center}

\begin{flushright}
    2024年1月24日

    全学学類・専門学群・総合学域群代表者会議 \; 決議
\end{flushright}

\vskip 2\baselineskip

\noindent
(目的)

第1条 \;
この規程は、筑波大学学生の活動に関する法人規程第2条に定める学生団体、学長決定等に定める学生組織その他の学内で活動する大学から認定を受けた団体(以下、「学生団体等」という。)を明確にするとともに、筑波大学における新入生歓迎活動及び勧誘活動(以下、「新歓活動」という。)が安全かつ公平に行われることを目的とし、同法人規程の下で定める。

\vskip\baselineskip

\noindent
(対象)

第2条 \;
すべての学生は、学生団体等の新歓活動に際しこの規則に従わなければならない。

2 \;
新歓活動は、学生が学生団体等を代表してこれを行う。

\vskip\baselineskip

\noindent
(新歓活動)

第3条 \;
新歓活動とは、自身の所属する学生団体等を新入生に紹介し、新たな構成員を確保することを目的とした集会、掲示、広報などの活動全般をいう。

\vskip\baselineskip

\noindent
(新歓期間)

第4条 \;
入学式の開催日から5月末日までを、新入生歓迎期間(以下、「新歓期間」という。)と呼称する。

\vskip\baselineskip

第5条 \;
すべての学生団体等は、新歓期間以前に新歓活動を行うことができない。

2 \;
前項の規定にかかわらず、以下に掲げる活動は、これを規制しない。
%
\begin{enumerate}
    \item SNS を通じた広報活動
    \item その他新歓委が別に定める活動
\end{enumerate}

\vskip\baselineskip

\noindent
(新入生歓迎委員会)

第6条 \;
新歓期間の運営は、文化系サークル連合会、体育会、芸術系サークル連合会(以下、「三系」という。)と全学学類・専門学群・総合学域群代表者会議(以下、「全代会」という。)の構成員から組織される新入生歓迎委員会(以下、「新歓委」という。)がこれ行う。

\vskip\baselineskip

\noindent
(新入生歓迎祭)

第7条 \;
新入生歓迎祭(以下、「新歓祭」という。)は、入学式が挙行される日の翌日以降における最初の土曜日に、新歓委がこれを開催する。

2 \;
新歓祭の運営は学園祭実行委員会がこれを行う。

\vskip\baselineskip

第8条 \;
すべての学生団体等は、新歓祭に参加することができる。

2 \;
学生団体等は、新歓祭の参加に際し学園祭実行委員会へ申請しなければならない。

3 \;
一般学生団体は、新歓祭の参加に際し運営費として新歓委の定める金額を納入しなければならない。

\vskip\baselineskip

\noindent
(掲示・配布)

第9条 \;
すべての学生団体等は、新歓活動としての広報物の掲示又は配布に際し、新歓委が別に定める手続きを行わなければならない。

\vskip\baselineskip

\noindent
(活動場所)

第10条 \;
学内における新歓活動は、新歓委が定める活動可能場所にてこれを行う。

\vskip\baselineskip

\noindent
(許可証の携帯)

第11条 \;
前条に基づく新歓活動を行う際は、新歓委の発行する許可証を携帯しなければならない。

\vskip\baselineskip

\noindent
(三系所属団体の優先権)

第12条 \;
三系に属する課外活動団体は、新歓期間の活動において他の学生団体等に対し優先権を行使することができる。

2 \;
前項に定める優先権は特設掲示板の利用など、すべての学生団体等が公平に掲示・活動することが困難な事柄に行使でき、対象となる新歓活動は新歓委が定める。

\vskip\baselineskip

\noindent
(新誘活動における禁止事項)

第13条 \;
新歓活動において以下に掲げる事項を行ってはならない。
%
\begin{enumerate}
    \item 法令、大学の定める諸規則その他公序良俗に違反する行為
    \item 新入生を身体的、心理的に傷つける行為
    \item 新入生に対する強引な勧誘行為
    \item 学生団体等を代表せずに行う新歓活動に類する行為
    \item 他団体の新歓活動を妨害する行為
    \item 特定の政党若しくは宗教団体に係る活動並びにこれらに誘導する行為
    \item 商業活動及びこれに誘導する行為
    \item 学内における指定場所以外での新歓活動
    \item 新歓委の管理するWebページへの、著作権の侵害、特定の個人若しくは団体の誹謗中傷その他公序良俗に違反する書き込み
    \item 学外者による勧誘行為
    \item 本人の所属する団体以外への勧誘行為
    \item その他新歓委が不適切であると判断する行為
\end{enumerate}

\vskip\baselineskip

\noindent
(宿舎エリアでの勧誘)

第14条 \;
宿舎エリアにおける新歓活動は、新歓委が別に定める事項に則り、これを行う。

2 \;
宿舎エリアは、平砂、追越、一の矢、春日の宿舎周辺等をいう。

\vskip\baselineskip

第15条 \;
宿舎エリアでの新歓活動においては、第十三条各号に定める事項に加え、以下に掲げる事項を行ってはならない。
%
\begin{enumerate}
    \item 宿舎内への立ち入り
    \item 駐車場及び駐輪場以外への無許可での駐車や駐輪
    \item 宿舎放送の使用
    \item 午後9時から午前8時までの間の新歓活動
    \item 騒音など住民に迷惑をかける行為
    \item その他新歓委が不適切であると判断する行為
\end{enumerate}

\vskip\baselineskip

\noindent
(細則)

第16条 \;
その他必要な事項は新歓委が別に定める。

\vskip\baselineskip

\noindent
(罰則)

第17条 \;
この規程又は新歓委が別に定める事項に違反する行為を確認したとき、その違反の規模、確認した事実を新歓委及び大学と共有する。

2 \;
新歓委は、確認された違反の内容に応じて、当該学生団体等に罰則を与える。

3 \;
違反の確認は、新歓委及び新歓委が委託する者がこれを行う。

4 \;
罰則の方法は、新歓委が別に定める。

\vskip\baselineskip

\noindent
(改廃)

第18条 \;
この規制の改正及び廃止は、全代会の承認による。

2 \;
この規程を改正又は廃止する議案は、課外活動団体会議及び各系別責任者会議並びに全代会の合意の下に発議される。

\section*{附則}

本規定は、令和6年1月24日から施行する。

\end{document}
